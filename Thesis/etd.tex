%
% PROJECT: <ETD> Electronic Thesis and Dissertation Initiative
%   TITLE: LaTeX report template for ETDs in LaTeX
%  AUTHOR: Neill Kipp, nkipp@vt.edu
%     URL: http://etd.vt.edu/latex/
% SAVE AS: etd.tex
% REVISED: September 6, 1997 [GMc 8/30/10]
% 

% Instructions: Remove the data from this document and replace it with your own,
% keeping the style and formatting information intact.  More instructions
% appear on the Web site listed above.

%\documentclass[12pt,dvips]{report}
\documentclass[12pt]{report}
\usepackage{bm}
\newenvironment{MyFigure}[1][]{\begin{figure}[#1]\vspace{1.0cm}}{\vspace{1.0cm}\end{figure}}
%\usepackage[dvips]{graphicx}
\usepackage{graphicx}
\newcommand{\overbar}[1]{\mkern 2.2mu\overline{\mkern-2.2mu#1\mkern-2.2mu}\mkern 2.2mu}

\setlength{\textwidth}{6.5in}
\setlength{\textheight}{8.5in}
\setlength{\evensidemargin}{0in}
\setlength{\oddsidemargin}{0in}
\setlength{\topmargin}{0in}

\setlength{\parindent}{0pt}
\setlength{\parskip}{0.1in}

% Uncomment for double-spaced document.
\renewcommand{\baselinestretch}{2}

\usepackage{epsf}

\begin{document}

\thispagestyle{empty}
\pagenumbering{roman}
\begin{center}

% TITLE
{\Large 
Toward a Reduced-Scaling Method for Calculating Coupled Cluster Response Properties
}

\vfill

Ashutosh Kumar

\vfill

Dissertation submitted to the Faculty of the \\
Virginia Polytechnic Institute and State University \\
in partial fulfillment of the requirements for the degree of

\vfill

Doctor of Philosophy \\
in \\
Literature and Technology

\vfill

T. Daniel Crawford, Chair \\
Eduard Valeyev \\
Diego Troya \\
Alan Esker

\vfill

May 9, 2018 \\
Blacksburg, Virginia

\vfill

Keywords: Coupled Cluster, Reduced-Scaling, Response Properties
\\
Copyright 2018, Ashutosh Kumar

\end{center}

\pagebreak

\thispagestyle{empty}
\begin{center}

{\large 
Toward a Reduced-Scaling Method For Calculating Coupled Cluster Response Properties
}

\vfill

Ashutosh Kumar

\vfill

(ABSTRACT)

\vfill

\end{center}
One of the central problems limiting the application of accurate {\em ab
initio} methods to large molecular systems is their high computational costs,
i.e., their computing and storage requirements exhibit polynomial scaling with
the size of the system.  For example, the coupled cluster method --- the
``gold standard'' of quantum chemistry --- scales as ${\cal O}(N^6)$, where
$N$ refers to the number of molecular orbitals (MOs). 

\vfill

% GRANT INFORMATION
This work was supported by a grant (CHE-1465149) from
the U.S. National Science Foundation. Advanced Research Computing 
Center at Virginia Tech provided the necessary computational resources and technical 
support for all the calculations reported here.
%That this work received support from the Southeastern Universities
%Research Association (SURA) ``Monticello Library Project'' is purely
%coincidental.

\pagebreak

% Dedication and Acknowledgments are both optional
% \chapter*{Dedication}
 \chapter*{Acknowledgments}

\tableofcontents
\pagebreak

\listoffigures
\pagebreak

\listoftables
\pagebreak

\pagenumbering{arabic}
\pagestyle{myheadings}

\chapter{Introduction}
\markright{Ashutosh Kumar \hfill Chapter 1. Introduction \hfill}

One of the central problems limiting the application of accurate {\em ab
initio} methods to large molecular systems is their high computational costs,
i.e., their computing and storage requirements exhibit polynomial scaling with
the size of the system.  For example, the coupled cluster method --- the
``gold standard'' of quantum chemistry --- scales as ${\cal O}(N^6)$, where
$N$ refers to the number of molecular orbitals (MOs). 
Over the last half
century, numerous techniques for the reduction of the size of the MO space
have been introduced to to overcome this scaling obstacle
 and L{\"o}wdin’s
introduction\cite{Lowdin55} of ``natural orbitals'' (NOs) as eigenvectors of
the one-electron reduced density matrix (1RDM) in 1955 stands as one of the
earliest works in this area. The frozen virtual NO (FVNO) scheme has been
successfully applied within coupled cluster theory for calculating correlation
energies, ionization potentials etc., where truncations of even up to 50\% of
the virtual space has been shown to introduce minimal
errors.\cite{Landau10,Mester17}  In the FVNO approach, the NOs are obtained
from the 1RDM of a less expensive correlated method such as second-order M\o
ller-Plesset (MP2) theory, and the virtual space is truncated based on the
orbitals' corresponding occupation numbers.

My research in the Crawford group at Virginia Tech has primarily focused on
the first successful extension of the FVNO scheme to calculate higher-order
response properties like dynamic polarizabilities and specific optical
rotations within the coupled cluster linear-response formalism. The
conventional FVNO method performs poorly for these properties, yielding large
errors which increase linearly (at best) with the number of frozen virtual
orbitals.\cite{Kumar17} My investigation revealed that the source of these
errors is the removal of diffuse virtual orbitals --- NOs that contribute
little to the correlation energy (and thus have low occupation numbers), but
are vital to the description of the wave function responses associated with
polarizabilities and related properties.  Indeed, applying the FVNO procedure
only to the non-diffuse virtual space while retaining the diffuse MOs in the
canonical basis itself greatly minimizes these errors.  However, the number 
of such orbitals that must be retained could be very high for larger basis
sets, thus severaly limiting this approach for large molecules.  

Based on these observations, I have developed a technique which I call FVNO++,
where instead of the ground state MP2-1RDM, a second-order perturbed density
based on a limited number of iterations of the second-order coupled cluster
(CC2) method is used to obtain perturbed ``natural orbitals''. A proof of
concept study where we chose the leading contribution to the full second-order
perturbed CCSD density as our 1RDM, produced less than 1\% errors even after
removing close to 50\% of the virtual space in most of the test cases, thus
validating this approach. The initial results obtained using guess densities
based on CC2 used in conjunction with corrections for the external truncated
space have been very promising and more results are forthcoming.
I will now extend this approach to the local pair natural orbital (LPNO)
methods, where a separate 1RDM is defined for every occupied pair of MOs
leading to separate non-orthogonal virtual spaces. The usual LPNO approach
suffers from the same deficiencies as the FVNO method described above and hence
performs poorly for such properties.\cite{McAlexander15:LRCC}
\chapter{Modeling Optical Rotation}
\chapter{Coupled Cluster Theory}
\subsection{Ground State}
\subsection{Response Theory}
\subsubsection{Sum of States}
\subsubsection{Linear Response}
\chapter{Frozen Virtual Natural Orbitals for Coupled-Cluster Linear-Response Theory}
\input{FVNO.tex}
\chapter{Frozen Virtual Natural Orbitals++ for Coupled-Cluster Linear-Response Theory}
\input{FVNO++.tex}
\chapter{PNO++ approach for Coupled-Cluster Linear-Response Theory}
\input{PNO++.tex}
%%%%%%%%%%%%%%%%%
%
% Include an EPS figure with this command:
%   \epsffile{filename.eps}
%

%%%%%%%%%%%%%%%%
%
% Do tables like this:

% \begin{table}
% \caption{The Graduate School wants captions above the tables.}
%\begin{center}
% \begin{tabular}{ccc}
% x & 1 & 2 \\ \hline
% 1 & 1 & 2 \\
% 2 & 2 & 4 \\ \hline
% \end{tabular}
%\end{center}
% \end{table}

%%%%%%%%%%%%%%%%%%%%%%%%%%%%%%%%

% If you are using BibTeX, uncomment the following:
%\thebibliography
%\begin{thebibliography}{99}
%\bibitem{Shavitt09}I. Shavitt and R. J. Bartlett. Many-Body Methods in Chemistry and
%Physics: MBPT and Coupled-Cluster Theory; Cambridge University
%Press: Cambridge, 2009. 
%\end{thebibliography}
\bibliographystyle{unsrt}
\bibliography{refs}
%
% Otherwise, uncomment the following:
%\chapter*{Bibliography}

% \appendix

% In LaTeX, each appendix is a "chapter"
% \chapter{Program Source}
\end{document}
